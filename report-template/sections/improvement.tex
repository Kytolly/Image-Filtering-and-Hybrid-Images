
基于本次实验的实现和体验,以下是一些对实验方法或代码的改进建议:

\begin{enumerate}
    \item 优化 \texttt{my\_imfilter} 函数的效率:
    当前 \texttt{my\_imfilter} 函数采用的是直接的空间域卷积实现,
    对于较大的图像或滤波器,计算开销较大。可以考虑引入更高效的卷积方法,比如在频域上使用快速傅里叶变换;

    \item 增强 \texttt{gen\_hybrid\_image} 函数的灵活性:
    目前的 \texttt{gen\_hybrid\_image} 函数硬编码使用了高斯滤波器进行高低通滤波。
    可以改进为允许用户选择或传入不同类型的滤波器,
    以探索不同滤波器对混合图像效果的影响。

    \item 边界处理方式的多样化:
    \texttt{my\_imfilter} 中只实现了反射填充。可以增加对其他边界填充方式的支持,
    如零填充(zero padding)、边缘复制填充(replicate padding)等,
    并在实验中比较不同填充方式对滤波结果,特别是图像边缘的影响。

\end{enumerate}
