\section{核心代码}

如代码 1 所示。

\begin{lstlisting}[caption={student.py}, label={lst:code-example}, captionpos=t, language=python]
    # Project Image Filtering and Hybrid Images Stencil Code
    # Based on previous and current work
    # by James Hays for CSCI 1430 @ Brown and
    # CS 4495/6476 @ Georgia Tech
    import numpy as np
    from numpy import pi, exp, sqrt
    from skimage import io, img_as_ubyte, img_as_float32
    from skimage.transform import rescale
    import math
    
    def my_imfilter(image, filter):
      """
      Your function should meet the requirements laid out on the project webpage.
      Apply a filter to an image. Return the filtered image.
      Inputs:
      - image -> numpy nd-array of dim (m, n, c)
      - filter -> numpy nd-array of odd dim (k, l)
      Returns
      - filtered_image -> numpy nd-array of dim (m, n, c)
      Errors if:
      - filter has any even dimension -> raise an Exception with a suitable error message.
      """

      image_height, image_width = image.shape[:2]
      filter_height, filter_width = filter.shape
      num_channels = 1 if image.ndim == 2 else image.shape[2]
    
      # Calculate padding amounts
      pad_height = filter_height // 2
      pad_width = filter_width // 2
      padded_image = np.pad(image, ((pad_height, pad_height), (pad_width, pad_width), (0, 0) if num_channels > 1 else (0, 0)), mode='reflect')
    
      # filter process: convolution
      filtered_image = np.zeros_like(image)
      flipped_filter = np.flip(filter, axis=(0, 1))
      for c in range(num_channels):
          for i in range(image_height):
              for j in range(image_width):
                  image_patch = padded_image[i:i+filter_height, j:j+filter_width, c] if num_channels > 1 else padded_image[i:i+filter_height, j:j+filter_width]
                  
                  filtered_image[i, j, c] = np.sum(image_patch * flipped_filter) if num_channels > 1 else np.sum(image_patch * flipped_filter)
      return filtered_image
    
    
    def gen_hybrid_image(image1, image2, cutoff_frequency):
      """
       Inputs:
       - image1 -> The image from which to take the low frequencies.
       - image2 -> The image from which to take the high frequencies.
       - cutoff_frequency -> The standard deviation, in pixels, of the Gaussian
                             blur that will remove high frequencies.
    
       Task:
       - Use my_imfilter to create 'low_frequencies' and 'high_frequencies'.
       - Combine them to create 'hybrid_image'.
      """
    
      assert image1.shape[0] == image2.shape[0]
      assert image1.shape[1] == image2.shape[1]
      assert image1.shape[2] == image2.shape[2]
    
      # Steps:
      # (1) Remove the high frequencies from image1 by blurring it. The amount of
      #     blur that works best will vary with different image pairs
      # generate a 1x(2k+1) gaussian kernel with mean=0 and sigma = s, see https://stackoverflow.com/questions/17190649/how-to-obtain-a-gaussian-filter-in-python
      s, k = cutoff_frequency, int(cutoff_frequency*2)
      probs = np.asarray([exp(-z*z/(2*s*s))/sqrt(2*pi*s*s) for z in range(-k,k+1)], dtype=np.float32)
      kernel = np.outer(probs, probs)
      
      # Your code here:
      # low_frequencies = None # Replace with your implementation
      low_frequencies = my_imfilter(image1, kernel)
    
      # (2) Remove the low frequencies from image2. The easiest way to do this is to
      #     subtract a blurred version of image2 from the original version of image2.
      #     This will give you an image centered at zero with negative values.
      # Your code here #
      # high_frequencies = None # Replace with your implementation
      image2_low_frequencies = my_imfilter(image2, kernel)
      high_frequencies = image2 - image2_low_frequencies
    
      # (3) Combine the high frequencies and low frequencies
      # Your code here #
      # hybrid_image = None
      hybrid_image = low_frequencies + high_frequencies
    
      # (4) At this point, you need to be aware that values larger than 1.0
      # or less than 0.0 may cause issues in the functions in Python for saving
      # images to disk. These are called in proj1_part2 after the call to 
      # gen_hybrid_image().
      # One option is to clip (also called clamp) all values below 0.0 to 0.0, 
      # and all values larger than 1.0 to 1.0.
      hybrid_image = np.clip(hybrid_image, 0, 1)

      return low_frequencies, high_frequencies, hybrid_image
    
    def vis_hybrid_image(hybrid_image):
      """
      Visualize a hybrid image by progressively downsampling the image and
      concatenating all of the images together.
      """
      scales = 5
      scale_factor = [0.5, 0.5, 1]
      padding = 5
      original_height = hybrid_image.shape[0]
      num_colors = 1 if hybrid_image.ndim == 2 else 3
    
      output = np.copy(hybrid_image)
      cur_image = np.copy(hybrid_image)
      for scale in range(2, scales+1):
        # add padding
        output = np.hstack((output, np.ones((original_height, padding, num_colors),
                                            dtype=np.float32)))
        # downsample image
        cur_image = rescale(cur_image, scale_factor, mode='reflect')
        # pad the top to append to the output
        pad = np.ones((original_height-cur_image.shape[0], cur_image.shape[1],
                       num_colors), dtype=np.float32)
        tmp = np.vstack((pad, cur_image))
        output = np.hstack((output, tmp))
      return output
    
    def load_image(path):
      return img_as_float32(io.imread(path))
    
    def save_image(path, im):
      return io.imsave(path, img_as_ubyte(im.copy()))
    
\end{lstlisting}