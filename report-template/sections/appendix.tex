\section{代码示例}

示例代码如代码 \ref{lst:code-example} 所示。

\begin{lstlisting}[caption={总控程序main.rs}, label={l:code-example}, captionpos=t, language=rust]
  pub mod prep;
  pub mod lex;
  pub mod env;
  pub mod parse;
  
  use prep::Preprocessor;
  use lex::Lexer;
  use parse::Parser;
  use env::Env;
  
  fn main() {
      let path = "test/6";
      let mode = "file";
      let preprocessor = Preprocessor::new(path);
      println!("---------------------------------");
  
      let mut lexer = Lexer::new(preprocessor, path, mode);
      lexer.analyse();
      lexer.save();
      println!("---------------------------------");
  
      let mut env = Env::new();
      let s = lexer.get_stream();
      
      let mut parser = Parser::new(s, mode, path.to_string());
      match parser.analyse(&mut env) {
          Ok(()) => println!("compilered!"),
          _ => println!("syntax error!")
      }
  }
\end{lstlisting}

\begin{lstlisting}[caption={预处理器prep.rs}, label={2:code-example}, captionpos=t, language=rust]
  use std::fs::File;
  use std::io::{Read};
  
  pub struct Preprocessor {
      pub path: String, // 源程序名
      pub content: String, // 源程序字符流
  }
  
  impl Preprocessor {
      pub fn new(name: &str) -> Self{
          let mut p = Preprocessor{
              path: name.to_string() + ".pas",
              content: String::new(),
          };
          let mut input = File::open(&p.path).unwrap();
          input.read_to_string(&mut p.content).unwrap();
          println!("{}", p.content);
          p
      }
  }
  
\end{lstlisting}

\begin{lstlisting}[caption={词法分析器lex.rs}, label={3:code-example}, captionpos=t, language=rust]

\end{lstlisting}

\begin{lstlisting}[caption={语法分析器parse.rs}, label={4:code-example}, captionpos=t, language=rust]
  use crate::env::{Token, ErrorMessage, Env};
  use std::fs;
  use std::io::{Write};
  
  pub struct Parser {
      // LL1语法分析器,基于递归下降办法
      pub stream: Vec<Token>, // 输入的token流
      pub pos: usize, //当前token所在位置
      pub line: usize, // 当前token所在行数
      mode: &'static str, // 错误的打印模式
      name: String,
  }
  
  impl Parser {
      pub fn new(s: Vec<Token>, mode: &'static str, name: String) -> Self {
          let p = Parser {
              stream: s,
              pos: 0,
              line: 1,
              mode: mode,
              name: name,
          };
          p
      }
      pub fn analyse(&mut self, env: &mut Env) -> Result<(), ErrorMessage> {
          self.parse_node_program(env)
      }
  
      fn debug(&self) {
          println!("-------------------------");
          println!("debug point");
          println!("pos:{:?}", self.pos);
          println!("line:{:?}", self.line);
          println!("token:{:?}", self.current_token());
      }
      fn current_token(&self) -> Token {
          if self.pos >= self.stream.len() {
              Token::Eof
          } else {
              self.stream[self.pos].clone()
          }
      }
      fn advance(&mut self) {
          if self.pos >= self.stream.len() {
              return;
          }
          
          self.pos += 1;
          let mut tk = self.current_token();
          
          // 处理换行符
          while tk == Token::Eol && self.pos < self.stream.len() {
              self.line += 1;
              self.pos += 1;
              tk = self.current_token();
          }
      }
      fn match_token(&self, tk: Token) -> bool {
          self.current_token() == tk
      }
      fn error(&self, errmsg: ErrorMessage) {
          match self.mode {
              "console" => self.console_error(errmsg),
              "file" => self.file_error(&errmsg),
              _ => println!("invalid mode!"),
          };
      }
      fn file_error(&self, errmsg:&ErrorMessage) {
          let path = format!("{}.err", self.name);
          let mut file = fs::OpenOptions::new()
              .write(true)
              .append(true)
              .create(true)
              .open(&path)
              .expect("Failed to create error file");
          let err_msg = match errmsg {
              ErrorMessage::SyntaxError => format!("LINE{:?}: unknown token!\n", self.line),
              ErrorMessage::WrongReserveYouMeanFunction => format!("LINE{:?}: wrong reserve: you mean 'function'?\n", self.line),
              ErrorMessage::WrongReserveYouMeanRead => format!("LINE{:?}: wrong reserve: you mean 'read'?\n", self.line),
              ErrorMessage::WrongReserveYouMeanWrite => format!("LINE{:?}: wrong reserve: you mean 'write'?\n", self.line),
              ErrorMessage::WrongAssignToken => format!("LINE{:?}: wrong assign operator: you mean ':='?\n", self.line),
              ErrorMessage::InvalidTypeExpectedInterger => format!("LINE{:?}: invalid type: expected INTEGER\n", self.line),
              ErrorMessage::InvalidNumber => format!("LINE{:?}: Invalid number!\n", self.line),
              ErrorMessage::OverflowIdentifier => format!("LINE{:?}: Identifier length overflow!\n", self.line),
              ErrorMessage::FailMatchingSemicolon => format!("LINE{:?}: Semicolon matching failed!\n", self.line),
              ErrorMessage::MissingSemicolon => format!("LINE{:?}: missing a ';' at the end of the statement\n", self.line),
              ErrorMessage::MissingLeftParenthesis => format!("LINE{:?}: expected '(' following the function statement\n", self.line),
              ErrorMessage::MissingRightParenthesis => format!("LINE{:?}: expected ')' to cover the block\n", self.line),
              ErrorMessage::MissingIf => format!("LINE{:?}: expected 'if' \n", self.line),
              ErrorMessage::MissingThen => format!("LINE{:?}: expected 'then' \n", self.line),
              ErrorMessage::MissingElse => format!("LINE{:?}: expected 'else' \n", self.line),
              ErrorMessage::MissingMultiply => format!("LINE{:?}: expected '*' \n", self.line),
              ErrorMessage::SyntaxErrorExpectedABlock => format!("LINE{:?}: syntax error, expected a block\n", self.line),
              ErrorMessage::FailMatching => format!("LINE{:?}: Symbol matching error!\n", self.line),
              ErrorMessage::MissingEnd => format!("LINE{:?}: missing END: this block is not covered\n", self.line),
              ErrorMessage::ExpectedIdentifier => format!("LINE{:?}: Expected identifier in this field\n", self.line),
              ErrorMessage::FoundRepeatDeclarationInThisField => format!("LINE{:?}: this symbol's declaration repeated in this field\n", self.line),
          };
          file.write_all(err_msg.as_bytes()).expect("Failed to write error file");
      }
      fn console_error(&self, errmsg: ErrorMessage) {
          // 抛出错误
          // 这里还是简化实现
          // 应该扔到标准错误流中,写入文件
          // 不能和标准输出流混合
          // self.debug();
          match errmsg {
              ErrorMessage::SyntaxError => {
                  println!("LINE{:?}: syntax error, exited", self.line);
              }
              ErrorMessage::WrongReserveYouMeanFunction => {
                  println!("LINE{:?}: wrong reserve: you mean 'function'?", self.line);
              }
              ErrorMessage::WrongReserveYouMeanRead => {
                  println!("LINE{:?}: wrong reserve: you mean 'read'?", self.line);
              }
              ErrorMessage::WrongReserveYouMeanWrite => {
                  println!("LINE{:?}: wrong reserve: you mean 'write'?", self.line);
              }
              ErrorMessage::WrongAssignToken => {
                  println!("LINE{:?}: wrong assign operator: you mean ':='?", self.line);
              }
              ErrorMessage::InvalidTypeExpectedInterger => {
                  println!("LINE{:?}: invalid type here, expected integer", self.line);
              }
              ErrorMessage::InvalidNumber => {
                  println!("LINE{:?}: invalid number", self.line);
              }
              ErrorMessage::OverflowIdentifier => {
                  println!("LINE{:?}: the length of identifier overflows", self.line);
              }
              ErrorMessage::FailMatchingSemicolon => {
                  println!("LINE{:?}: fail in matching semicolon", self.line);
              }
              ErrorMessage::MissingSemicolon => {
                  println!("LINE{:?}: missing a ';' at the end of the statement", self.line);
              }
              ErrorMessage::MissingLeftParenthesis => {
                  println!("LINE{:?}: missing a '(' following the function statement", self.line);
              }
              ErrorMessage::MissingRightParenthesis => {
                  println!("LINE{:?}: missing a ')' to cover the block", self.line);
              }
              ErrorMessage::MissingMultiply => {
                  println!("LINE{:?}: expected '*' ", self.line);
              }
              ErrorMessage::MissingIf => {
                  println!("LINE{:?}: expected 'if' ", self.line);
              }
              ErrorMessage::MissingThen => {
                  println!("LINE{:?}: expected 'then' ", self.line);
              }
              ErrorMessage::MissingElse => {
                  println!("LINE{:?}: expected 'else' ", self.line);
              }
              ErrorMessage::SyntaxErrorExpectedABlock => {
                  println!("LINE{:?}: syntax error, expected a block", self.line);
              }
              ErrorMessage::FailMatching => {
                  println!("LINE{:?}: 符号匹配错误!", self.line);
              }
              ErrorMessage::MissingEnd => {
                  println!("LINE{:?}: missing END: this block is not covered", self.line);
              }
              ErrorMessage::ExpectedIdentifier => {
                  println!("LINE{:?}: expected indentifier", self.line);
              }
              ErrorMessage::FoundRepeatDeclarationInThisField => {
                  println!("LINE{:?}: the declaration of this indentifier repeated in this scope", self.line);
              }
          }
          println!("-------------------------");
      }
      fn skip_bad_line(&mut self) {
          let mut tk = self.current_token();
          while tk != Token::Eol && tk != Token::Eof {
              self.advance();
              tk = self.current_token();
          }
      }
      fn handle_error(&mut self, errmsg: ErrorMessage) -> Result<(), ErrorMessage>{
          let res = Err(errmsg.clone());
          self.debug();
          self.error(errmsg);
          self.skip_bad_line();
          res
      }
      
      fn parse_node_program(&mut self, env: &mut Env) -> Result<(), ErrorMessage>{
          // <程序> → <分程序>
          self.parse_node_block(env)
      }
      fn parse_node_block(&mut self, env: &mut Env) -> Result<(), ErrorMessage>{
          // <分程序> → begin <说明语句表><执行语句表> end
          match self.match_token(Token::Begin) {
              true => self.advance(),
              false => return self.handle_error(ErrorMessage::SyntaxErrorExpectedABlock)
          }
          env.enter_scope();
          match self.parse_node_declaration_statement_table(env) {
              Ok(_) => (),
              Err(e) => return self.handle_error(e),
          }
          // match self.match_token(Token::Semicolon) {
          //     true => self.advance(),
          //     false => return self.handle_error(ErrorMessage::MissingSemicolon),
          // }
          match self.parse_node_execution_statement_table() {
              Ok(_) => (),
              Err(e) => return self.handle_error(e),
          }
          match self.match_token(Token::End) {
              true => self.advance(),
              false => return self.handle_error(ErrorMessage::MissingEnd)
          }
          env.exit_scope();
          Ok(())
      }
      fn parse_node_declaration_statement_table(&mut self, env:&mut Env) -> Result<(), ErrorMessage>{
          // <说明语句表> → {<说明语句> ;}
          loop {
              if !self.match_token(Token::Integer) {
                  // 检查FOLLOW 集
                  match self.current_token() {
                      Token::Read | Token::Write | Token::If | Token::Identifier(_) | Token::End | Token::Eof => {
                          return Ok(());
                      },
                      _ => return self.handle_error(ErrorMessage::SyntaxError),
                  }
              }
              
              // 匹配说明语句
              match self.parse_node_declaration_statement(env) {
                  Ok(_) => (),
                  Err(e) => return self.handle_error(e),
              }
              
              // 匹配分号
              match self.match_token(Token::Semicolon) {
                  true => self.advance(),
                  false => return self.handle_error(ErrorMessage::MissingSemicolon),
              }
          }
      }
      fn parse_node_declaration_statement(&mut self, env: &mut Env) -> Result<(), ErrorMessage>{
          // <说明语句> → integer <说明语句'>
          match self.match_token(Token::Integer) {
              true => self.advance(),
              false => return self.handle_error(ErrorMessage::InvalidTypeExpectedInterger)
          }
          self.parse_node_declaration_statement_prime(env)
      }
      fn parse_node_declaration_statement_prime(&mut self, env: &mut Env) -> Result<(), ErrorMessage>{
          // <说明语句'> → <变量> | function <标识符>(<参数>)<函数体> ;
          if self.match_token(Token::Function) {
              // 函数说明分支
              self.advance();
  
              // 获取函数标识符名称
              let pname = match self.parse_node_identifier() {
                  Ok(pname) => pname,
                  Err(e) => return self.handle_error(e),
              };
              // 检查是否重复声明,若没有则添加声明
              if env.check_repeat(pname.clone()){
                  return self.handle_error(ErrorMessage::FoundRepeatDeclarationInThisField);
              } else{
                  env.add_procedure(pname.clone());
              }
  
              match self.match_token(Token::LeftParenthesis) {
                  true => self.advance(),
                  false => return self.handle_error(ErrorMessage::MissingLeftParenthesis)
              }
              match self.parse_node_parameter() {
                  Ok(_) => (),
                  Err(e) => return self.handle_error(e),
              }
              match self.match_token(Token::RightParenthesis) {
                  true => self.advance(),
                  false => return self.handle_error(ErrorMessage::MissingRightParenthesis)
              }
              match self.match_token(Token::Semicolon) {
                  true => {
                      self.advance();
                      match self.parse_node_function_body(env) {
                          Ok(_) => Ok(()),
                          Err(e) => return self.handle_error(e),
                      }
                  },
                  false => self.handle_error(ErrorMessage::MissingSemicolon),
              }
          } else {
              // 变量说明分支
              // 获取变量名字
              let vname = match self.parse_node_variable() {
                  Ok(vname) => vname,
                  Err(e) => return self.handle_error(e),
              };
              // 检查是否重复声明,若没有则添加声明
              if env.check_repeat(vname.clone()){
                  return self.handle_error(ErrorMessage::FoundRepeatDeclarationInThisField);
              } else{
                  env.add_variable(vname.clone(), "F".to_string(),0);
                  Ok(())
              }
          }
      }
      fn parse_node_function_body(&mut self, env:&mut Env) -> Result<(), ErrorMessage>{
          // <函数体> → begin <说明语句表><执行语句表> end
          match self.match_token(Token::Begin) {
              true => self.advance(),
              false => return self.handle_error(ErrorMessage::SyntaxErrorExpectedABlock)
          }
          match self.parse_node_declaration_statement_table(env) {
              Ok(_) => (),
              Err(e) => return self.handle_error(e),
          }
          // match self.match_token(Token::Semicolon) {
          //     true => self.advance(),
          //     false => return self.handle_error(ErrorMessage::MissingSemicolon)
          // }
          match self.parse_node_execution_statement_table() {
              Ok(_) => (),
              Err(e) => return self.handle_error(e),
          }
          match self.match_token(Token::End) {
              true => {
                  self.advance();
                  Ok(())
              },
              false => return self.handle_error(ErrorMessage::MissingEnd)
          }
      }
      fn parse_node_parameter(&mut self) -> Result<(), ErrorMessage>{
          // <参数> → <算术表达式>
          match self.parse_node_expression() {
              Ok(_) => Ok(()),
              Err(e) => return self.handle_error(e),
          }
      }
      fn parse_node_execution_statement_table(&mut self) -> Result<(), ErrorMessage>{
          // <执行语句表> → {<执行语句> ;}
          // FOLLOW(<执行语句表>) 包含 'end' 和 '$'
          loop {
              // 检查当前token是否可以开始一个执行语句
              match self.current_token() {
                  Token::Read | Token::Write | Token::If | Token::Identifier(_) => {
                      // 可以开始执行语句,继续解析
                  },
                  Token::End | Token::Eof => {
                      // 否则,检查是否在 FOLLOW 集里 (end 或 EOF)
                      // 如果在,说明执行语句表结束
                      return Ok(());
                  },
                  _ => return self.handle_error(ErrorMessage::SyntaxError),
              }
              
              // 解析一个执行语句
              match self.parse_node_execution_statement() {
                  Ok(_) => (),
                  Err(e) => return self.handle_error(e),
              }
              
              // 期望匹配分号
              match self.match_token(Token::Semicolon) {
                  true => self.advance(),
                  false => return self.handle_error(ErrorMessage::MissingSemicolon),
              }
          }
      }
      fn parse_node_execution_statement(&mut self) -> Result<(), ErrorMessage>{
          // <执行语句> → <读语句>│<写语句>│<赋值语句>│<条件语句>
          match self.current_token() {
              Token::Read => self.parse_node_read_statement(),
              Token::Write => self.parse_node_write_statement(),
              Token::If => self.parse_node_conditional_statement(),
              Token::Identifier(_) => self.parse_node_assignment_statement(),
              _ => self.handle_error(ErrorMessage::SyntaxError)
          }
      }
      fn parse_node_assignment_statement(&mut self) -> Result<(), ErrorMessage>{
          // <赋值语句> → <变量> := <算术表达式>
          match self.parse_node_variable() {
              Ok(_) => (),
              Err(e) => return self.handle_error(e),
          }
          match self.match_token(Token::Assign) {
              true => self.advance(),
              false => return self.handle_error(ErrorMessage::WrongAssignToken)
          }
          match self.parse_node_expression() {
              Ok(_) => Ok(()),
              Err(e) => return self.handle_error(e),
          }
      }
      fn parse_node_conditional_statement(&mut self) -> Result<(), ErrorMessage>{
          // <条件语句> → if<条件表达式>then<执行语句>else <执行语句>
          match self.match_token(Token::If) {
              true => self.advance(),
              false => return self.handle_error(ErrorMessage::MissingIf)
          }
          match self.parse_node_condition() {
              Ok(_) => (),
              Err(e) => return self.handle_error(e),
          }
          match self.match_token(Token::Then) {
              true => self.advance(),
              false => return self.handle_error(ErrorMessage::MissingThen)
          }
          match self.parse_node_execution_statement() {
              Ok(_) => (),
              Err(e) => return self.handle_error(e),
          }
          match self.match_token(Token::Else) {
              true => self.advance(),
              false => return self.handle_error(ErrorMessage::MissingElse)
          }
          match self.parse_node_execution_statement() {
              Ok(_) => Ok(()),
              Err(e) => return self.handle_error(e),
          }
      }
      fn parse_node_read_statement(&mut self) -> Result<(), ErrorMessage>{
          // <读语句> → read(<变量>)
          match self.match_token(Token::Read) {
              true => self.advance(),
              false => return self.handle_error(ErrorMessage::WrongReserveYouMeanRead)
          }
          match self.match_token(Token::LeftParenthesis) {
              true => self.advance(),
              false => return self.handle_error(ErrorMessage::MissingLeftParenthesis)
          }
          match self.parse_node_variable() {
              Ok(_) => (),
              Err(e) => return self.handle_error(e),
          }
          match self.match_token(Token::RightParenthesis) {
              true => {
                  self.advance();
                  Ok(())
              },
              false => return self.handle_error(ErrorMessage::MissingRightParenthesis)
          }
      }
      fn parse_node_write_statement(&mut self) -> Result<(), ErrorMessage>{
          // <写语句> → write(<变量>)
          match self.match_token(Token::Write) {
              true => self.advance(),
              false => return self.handle_error(ErrorMessage::WrongReserveYouMeanWrite)
          }
          match self.match_token(Token::LeftParenthesis) {
              true => self.advance(),
              false => return self.handle_error(ErrorMessage::MissingLeftParenthesis)
          }
          match self.parse_node_variable() {
              Ok(_) => (),
              Err(e) => return self.handle_error(e),
          }
          match self.match_token(Token::RightParenthesis) {
              true => {
                  self.advance();
                  Ok(())
              },
              false => return self.handle_error(ErrorMessage::MissingRightParenthesis)
          }
      }
      fn parse_node_condition(&mut self) -> Result<(), ErrorMessage>{
          // <条件表达式> → <算术表达式><关系运算符><算术表达式>
          match self.parse_node_expression() {
              Ok(_) => (),
              Err(e) => return self.handle_error(e),
          }
          match self.parse_node_relational_operator() {
              Ok(_) => self.advance(),
              Err(e) => return self.handle_error(e),
          }
          match self.parse_node_expression() {
              Ok(_) => (),
              Err(e) => return self.handle_error(e),
          }
          Ok(())
      }
      fn parse_node_expression(&mut self) -> Result<(), ErrorMessage>{
          // <算术表达式> → <项> <算术表达式'>
          match self.parse_node_term() {
              Ok(_) => (),
              Err(e) => return self.handle_error(e),
          }
          match self.parse_node_expression_prime() {
              Ok(_) => (),
              Err(e) => return self.handle_error(e),
          }
          Ok(())
      }
      fn parse_node_expression_prime(&mut self) -> Result<(), ErrorMessage>{
          // <算术表达式'> → -<项> <算术表达式'> | ε
          if self.match_token(Token::Minus) {
              self.advance();
              match self.parse_node_term() {
                  Ok(_) => (),
                  Err(e) => return self.handle_error(e),
              }
              match self.parse_node_expression_prime() {
                  Ok(_) => (),
                  Err(e) => return self.handle_error(e),
              }
              Ok(())
          }else { 
              Ok(())
          }
      }
      fn parse_node_term(&mut self) -> Result<(), ErrorMessage>{
          // <项> → <因子> <项'>
          match self.parse_node_factor() {
              Ok(_) => (),
              Err(e) => return self.handle_error(e),
          }
          match self.parse_node_term_prime() {
              Ok(_) => (),
              Err(e) => return self.handle_error(e),
          }
          Ok(())
      }
      fn parse_node_term_prime(&mut self) -> Result<(), ErrorMessage>{
          // <项'> → *<因子> <项'> | ε
          if self.match_token(Token::Multiply) {
              self.advance();
              match self.parse_node_factor() {
                  Ok(_) => (),
                  Err(e) => return self.handle_error(e),
              }
              match self.parse_node_term_prime() {
                  Ok(_) => (),
                  Err(e) => return self.handle_error(e),
              }
              Ok(())
          }else {
              Ok(())
          }
      }
      fn parse_node_factor(&mut self) -> Result<(), ErrorMessage>{
          // <因子> → <标识符> <因子后缀> | <常数> | (<算术表达式>)
          match self.current_token() {
              Token::LeftParenthesis => {
                  self.advance();
                  match self.parse_node_expression() {
                      Ok(_) => (),
                      Err(e) => return self.handle_error(e),
                  }
                  match self.match_token(Token::RightParenthesis) {
                      true => {
                          self.advance();
                          Ok(())
                      },
                      false => self.handle_error(ErrorMessage::MissingRightParenthesis)
                  }
              },
              Token::IntegerLiteral(_) => self.parse_node_constant(),
              Token::Identifier(_) => {
                  match self.parse_node_identifier() {
                      Ok(_) => (),
                      Err(e) => return self.handle_error(e),
                  }
                  match self.parse_node_factor_suffix() {
                      Ok(_) => (),
                      Err(e) => return self.handle_error(e),
                  }
                  Ok(())
              },
              _ => self.handle_error(ErrorMessage::SyntaxError)
          }
      }
      fn parse_node_factor_suffix(&mut self) -> Result<(), ErrorMessage>{
          // <因子后缀> → (<参数>)| ε
          match self.match_token(Token::LeftParenthesis) {
              true => self.advance(),
              false => return Ok(()),
          }
          match self.parse_node_parameter() {
              Ok(_) => (),
              Err(e) => return self.handle_error(e),
          }
          match self.match_token(Token::RightParenthesis) {
              true => self.advance(),
              false => return self.handle_error(ErrorMessage::MissingRightParenthesis)
          }
          Ok(())
      }
      fn parse_node_relational_operator(&mut self) -> Result<(), ErrorMessage>{
          // <关系运算符> → <│<=│>│>=│=│<>
          match self.current_token() {
              Token::Equal => Ok(()),
              Token::NotEqual => Ok(()),
              Token::Less => Ok(()),
              Token::LessEqual => Ok(()),
              Token::Greater => Ok(()),
              Token::GreaterEqual => Ok(()),
              _ => self.handle_error(ErrorMessage::SyntaxError)
          }
      }
      fn parse_node_variable(&mut self) -> Result<String, ErrorMessage>{
          // <变量> → <标识符>
          match self.parse_node_identifier() {
              Ok(name) => Ok(name),
              Err(e) => return Err(e),
          }
      }
      fn parse_node_constant(&mut self) -> Result<(), ErrorMessage>{
          // <常量> → <整数>
          match self.current_token() {
              Token::IntegerLiteral(_) => {
                  self.advance();
                  Ok(())
              },
              _ => self.handle_error(ErrorMessage::InvalidNumber)
          }
      }
      fn parse_node_identifier(&mut self) -> Result<String, ErrorMessage>{
          // <标识符>
          match self.current_token() {
              Token::Identifier(name) => {
                  self.advance();
                  Ok(name.clone())
              },
              _ => Err(ErrorMessage::ExpectedIdentifier),
          }
      }
  }
\end{lstlisting}

\begin{lstlisting}[caption={符号表控制器env.rs}, label={5:code-example}, captionpos=t, language=rust]
  use std::collections::HashMap;

  #[derive(Clone, PartialEq, Debug)]
  pub enum Token {
      // 所有的记号
      // 标识符
      Identifier(String),
      
      // 字面量
      IntegerLiteral(i64),
  
      // 算术运算符
      Minus,
      Multiply,
      Assign,
  
      // 关系运算符
      Equal,
      NotEqual,
      Less,
      LessEqual,
      Greater,
      GreaterEqual,
  
      // 分界符
      Begin,
      End,
      LeftParenthesis,
      RightParenthesis,
      Semicolon,
  
      // 行末提示符
      Eol,
  
      // 文件末提示符
      Eof,
  
      // 关键字
      Integer,
      Function,
      If,
      Then,
      Else,
      Read,
      Write,
  
      // 非法字符
      Illegal(char),
  }
  
  #[derive(Clone)]
  pub enum ErrorMessage {
      // 所有的报错信息
      SyntaxError,// 语法错误
      WrongReserveYouMeanFunction, // wrong reserve: you mean 'function'?
      WrongReserveYouMeanRead, // wrong reserve: you mean 'read'?
      WrongReserveYouMeanWrite, // wrong reserve: you mean 'write'?
      WrongAssignToken, // wrong assign operator: you mean ':='?
      InvalidTypeExpectedInterger, // 非法的类型,expected integer
      InvalidNumber, // 非法数字串
      OverflowIdentifier,// 标识符长度溢出
      FailMatchingSemicolon, // 冒号不匹配
      MissingSemicolon, // 缺一个分号,
      MissingLeftParenthesis, // expected '(' following the function statement
      MissingRightParenthesis, // expected ')' to cover the block
      MissingIf, // expected 'if' 
      MissingThen, // expected 'then' 
      MissingElse, // expected 'else'
      MissingMultiply, // expected 'multiply'
      SyntaxErrorExpectedABlock, // expected a block
      FailMatching, // 符号匹配错误
      MissingEnd, // begin没有匹配的end
      ExpectedIdentifier, // 符号无声明
      FoundRepeatDeclarationInThisField, //符号重复声明
  }
  
  #[derive(Clone)]
  pub struct VariableItem {
      // 变量表项
      pub vname: String, // 变量名
      pub vproc: String, // 所属过程
      pub vkind: i32, // 0-变量,1-形参
      pub vlev: i32, // 变量所在层次
      // pub vadr: i32, // 相对于第一个变量在变量表中的位置
      pub vtype: Vec<i32>, // 变量类型 
  }
  impl VariableItem {
      pub fn new(vname: String, vproc: String, vkind: i32, vlev: i32)-> Self{
          let v = VariableItem {
              vname: vname,
              vproc: vproc,
              vkind: vkind,
              vlev: vlev,
              vtype: Vec::new(),
          };
          v
      }
  }
  
  #[derive(Clone)]
  pub struct ProcedureItem {
      // 过程表项
      pub pname: String, // 过程名
      pub ptype: Vec<i32>, //过程类型
      pub plev: i32, // 过程所在层次
      // pub fadr: i32, // 第一个变量在变量表里的位置
      // pub ladr: i32, // 最后一个变量在变量表中的位置
  }
  impl ProcedureItem {
      pub fn new(pname: String, plev: i32) -> Self {
          let p = ProcedureItem {
              pname: pname, 
              ptype: Vec::new(),
              plev: plev
          };
          p
      }
  }
  
  #[derive(Clone)]
  pub struct SymbolTable {
      // 符号表,每个作用域都应该对应一个符号表
      pub variables: HashMap<String, VariableItem>, // 变量表
      pub procedures: HashMap<String, ProcedureItem>, // 过程表
      pub level: i32, // 当前作用域层级
  }
  impl SymbolTable {
      pub fn new(level: i32) -> Self {
          let st = SymbolTable {
              variables: HashMap::new(),
              procedures: HashMap::new(),
              level: level,
          };
          st
      }
      pub fn get_level(&self) -> i32{
          self.level
      }
  }
  
  #[derive(Clone)]
  pub struct Env {
      // 符号表栈,管理顶层符号表随作用域变化
      pub stack: Vec<SymbolTable>,
  }
  impl Env {
      pub fn new() -> Self {
          let e = Env {
              stack: Vec::new(),
          };
          e
      }
      pub fn enter_scope(&mut self){
          // 进入作用域,移入一个空符号表
          let t = SymbolTable::new(self.stack.len() as i32);
          self.stack.push(t);
      }
      pub fn exit_scope(&mut self){
          // 退出作用域,移出栈顶符号表
          self.stack.pop();
      }
      pub fn add_variable(&mut self, vname: String, vproc: String, vkind: i32){
          // 声明一个变量
          let t: &mut SymbolTable = self.stack.last_mut().unwrap();
          let item = VariableItem::new(vname.clone(), vproc, vkind, t.get_level());
          t.variables.insert(vname, item);
      }
      pub fn delete_cariable(&mut self, vname: String){
          // 析构一个变量
          let t: &mut SymbolTable = self.stack.last_mut().unwrap();
          t.variables.remove_entry(&vname);
      }
      pub fn add_procedure(&mut self, pname: String) {
          // 声明一个过程
          let t: &mut SymbolTable = self.stack.last_mut().unwrap();
          let item = ProcedureItem::new(pname.clone(), t.get_level());
          t.procedures.insert(pname, item);
      }
      pub fn delete_procedure(&mut self, pname: String) {
          // 析构一个过程
          let t: &mut SymbolTable = self.stack.last_mut().unwrap();
          t.procedures.remove_entry(&pname);
      }
      pub fn find_symbol(&self, name: String) -> bool{
          // 自顶向下查找一个符号
          for s in self.stack.iter().rev() {
              if s.variables.get(&name).is_some()||s.procedures.get(&name).is_some() {
                  return true;
              }
          }
          false
      }
      pub fn check_repeat(&self, name: String) -> bool {
          // 检查当前作用域是否重复声明某符号
          let t = self.stack.last().unwrap(); 
          t.variables.contains_key(&name) || t.procedures.contains_key(&name)
      }
      pub fn save(&self){
          // 保存在.var文件
      }
  }
\end{lstlisting}