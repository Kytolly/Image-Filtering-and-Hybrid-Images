首先,我在图像滤波、卷积原理及图像频率分析方面加深了理解。
通过亲自动手实现 \texttt{my\_imfilter} 函数,
我不仅掌握了卷积操作如何通过滤波器提取图像特征的原理,
还更加清楚地认识到图像中的高频和低频信息分别代表什么,以及它们在图像感知中的作用。

其次,我在编程实践和问题解决的能力上有所提高。实验过程中,实现卷积操作需要仔细处理索引、边界填充(反射填充)以及多通道图像的处理,这些细节的实现锻炼了我严谨的编程习惯。遇到的潜在问题,例如数值溢出或边界处理不当,通过调试和查阅资料(如 \texttt{numpy.pad} 的使用),
我成功克服了这些困难,这显著提升了我分析和解决图像处理实际问题的能力。

此外,我还学习到了参数调优和结果分析的重要性。
在生成混合图像时,不同\texttt{cutoff\_frequency} 的选择直接影响了
高低频成分的分离效果和最终混合图像在不同距离下的视觉呈现。

总的来说,这次实验不仅巩固了我对图像处理基本理论知识的掌握,特别是频率域处理的概念,更重要的是,通过从零开始实现核心算法,极大地提升了我的实践操作能力、代码调试能力以及对复杂图像处理流程的端到端理解。同时,通过对不同参数效果的分析,也培养了我科学实验中必不可少的分析和总结能力。
